\documentclass[a4paper,12pt]{article}

%Encoding
\usepackage[T1]{fontenc}
\usepackage[utf8]{inputenc}

%Language
\usepackage[english]{babel}

%Math
\usepackage{mathtools}
\usepackage{amssymb}
\usepackage{amsthm}
\newtheorem{theorem}{Theorem}

%Font
\usepackage{erewhon}
\usepackage[utopia,smallerops]{newtxmath}
\usepackage{microtype}

%Line spacing
\usepackage{setspace}
\onehalfspacing

%Layout
%\usepackage[margin=2.5cm]{geometry}

%Quotes
\usepackage{csquotes}

%Literature
%\usepackage[backend=biber,style=apa]{biblatex}
%\DeclareFieldFormat{apacase}{#1} %Capitalization like Bib File
%\addbibresource{Literature/Literature.bib}

%Keywords and JEL-Codes
\providecommand{\keywords}[1]{\noindent\textbf{Keywords:} #1}
\providecommand{\JELClassification}[1]{\noindent\textbf{JEL Classification:} #1}

%Blindtext
\usepackage{lipsum}

%Graphics
\usepackage{graphicx}

%Tables
\usepackage{tabularx}
\usepackage{booktabs}
\usepackage{siunitx}
\newcolumntype{d}{S[
    input-open-uncertainty=,
    input-close-uncertainty=,
    parse-numbers = false,
    table-align-text-pre=false,
    table-align-text-post=false
 ]}

%TikZ
\usepackage{tikz}

%Hyperlinks
\usepackage{hyperref}

\author{Marcel Harald Wachter\thanks{Email: \href{mailto:s6mawach@uni-bonn.de}{s6mawach@uni-bonn.de}} \and Cristian Enrique Gutierrez Charris\thanks{Email: \href{mailto:s6crguti@uni-bonn.de}{s6crguti@uni-bonn.de}}}
\title{The Title of the Paper}

\begin{document}
\pagenumbering{roman}
\maketitle
\thispagestyle{empty}

\begin{singlespace}
\begin{abstract}

\end{abstract}
\end{singlespace}

\newpage
\pagenumbering{arabic}

\section{Introduction}
\label{sec:introduction}
Labor economics is concerned with studying the exchange of labor services for wages.
Naturally, this includes a wide range of topics.
One of these topics is the use of levers, available to policymakers, to intervene in the labor market.
Due to this, research in labor economics and policy are closely connected and labor economics often attempts to uncover causal effects of either policy interventions or changes in individual choice variables on the outcome of the labor market.

The attempt to uncover causal effects makes the use of microeconometrics necessary. Traditionally, this can be done by means such as regression or instrument variables among other methods. Developments in machine learning provide tools to support these methods. One such method is the least absolute shrinkage and selection operator (lasso). The aim of this paper is to evaluate the usefulness of lasso for control variable selection as well as for the choice of instruments for instrument variable (IV) estimation.

The remainder of the paper is organised as follows. \hyperref[sec:theory]{Section 2}


\section{Theory}
\label{sec:theory}

\section{Simulation}
\label{sec:simulation}

\section{Application}
\label{sec:application}

\section{Conclusion}
\label{sec:conclusion}

\end{document}